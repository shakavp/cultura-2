\documentclass[12pt, a4paper]{article}

\usepackage[brazil]{babel}
\usepackage[utf8]{inputenc}
\usepackage[T1]{fontenc}
\usepackage[pdftex]{hyperref}
\usepackage{graphicx}
\usepackage{amsmath}
%\usepackage{indentfirst}
\usepackage{fancyhdr}
\usepackage{setspace}

% Formatação
\topmargin -1.5cm
\oddsidemargin -0.04cm
\evensidemargin -0.04cm
\textwidth 16.59cm
\textheight 21.94cm 

%Cabeçalho e Rodape
%\pagestyle{empty}                     % Sem numero de paginas
\pagestyle{fancy}                         %cabecalhos e rodares
\fancyhead[RO,RE]{}
%\fancyhead[LO,LE]{MAC0332 - SI para grupos de pesquisa\\
%	Glossário}
%\fancyfoot[LO,LE]{Confidential}
%\fancyfoot[RO,RE] {\thepage}
%\fancyfoot[CO,CE]{Grupo 3, 2011}

\parskip 7.2pt                        % Espaço entre paragrafos 7.2
\renewcommand{\baselinestretch}{1.5} % Espaçamento entre linhas = 1.5
%\parindent 0pt

% Tirar hifenização
\hyphenpenalty = 5000
\tolerance = 1000
\sloppy

\title{Universidade De São Paulo\\
    Faculdade De Filosofia, Letras e Ciências Humanas\\
    Departamento De Línguas Orientais\\
    \vspace{3cm}
    \textbf{\Huge{Período Azuchi-Momoyama}}\\
    \vspace{4cm}}    
\author{Docente: Prof. Dr. Koichi Mori\\
    Discente: Geraldo Castro Zampoli\\
    Nº USP 6552380}
\date{Dezembro de 2011}
%\date{\today}

\begin{document}

    \maketitle
    \newpage
    
    \section {Processo de Unificação por Nobunaga e Hideyoshi}
		\subsection {Sobre o Período}
			O período Azuchi-Momoyama começou ao fim do período Sengoku no Japão, foi nesse período em que a politica de unificação do shogunato Tokugawa ocorreu. Esse período se extendeu aproximadamente entre os anos 1568 à 1603, sob a liderança de Oda Nobunaga e seu sucessor, Toyotomi Hideyoshi. Este período recebe o nome do castelo de Nobunaga, o Castelo de Azuchi.	\\
			\\
			\indent O período Azuchi-Momoyama se inicia com a entrada de Nobunaga em Kyoto no ano de 1568, quando conduziu seu exército a capital imperial para instalar Ashikaga Yoshiaki como o décimo quinto e último \textit{shogun} do shogunato Ashikaga até a chegada ao poder de Tokugawa Ieyasu depois de sua vitória com o apoio do clã Toyotomi na Batalha de Sekigahara em 1600.\\

        \subsection {Oda Nobunaga}
            Oda Nobunaga (1534 - 1582) foi um proeminente \textit{daimyo} (senhor feudal) do período Sengoku ao período Azuchi-Momoyama da história japonesa. Filho de um \textit{daimyo} menor da província de Owari, depois da morte de seu pai lutou contra outros membros de sua família para o controle do clã, matando um de seus irmãos no processo.\\
            \\
            \indent Durante a última metade do século 16, um número de diferentes \textit{daimyo} se tornoram fortes o suficiente seja para manipular o \textit{bakufu} Muromachi em proveito próprio ou derrubá-lo completamente. Uma tentativa de derrubar o \textit{bakufu} foi feita em 1560 por Imagawa Yoshimoto, cuja marcha para a capital chegou ao fim nas mãos de Oda Nobunaga na Batalha de Okehazama. Nesta batalha, Oda Nobunaga derrotou Imagawa Yoshimoto e se estabeleceu como um dos senhores da guerra no período Sengoku, também foi depois dessa batalha que Tokugawa Ieyasu da Província de Mikawa se juntou o Oda, sendo um fiel aliado deste momento até sua morte. Em 1562, o clã Tokugawa que estava ao leste do território de Nobunaga se tornou independente do clã Imagawa, e aliado com Nobunaga. A parte oriental do território de Nobunaga não foi invadida por esta aliança. Em 1565, uma aliança de Matsunaga e dos clãs Miyoshi tentou um golpe com o assassinato de Ashikaga Yoshiteru, o décimo terceiro \textit{shogun} Ashikaga. Disputas internas, no entanto, impediu-os de agir rapidamente para legitimar sua reivindicação de poder, somente em 1568 que conseguiram instalar o primo do Yoshiteru, Ashikaga Yoshihide, como o \textit{shogun} seguinte. O fracasso em entrar em Kyoto e ganhar reconhecimento da corte imperial, fez com que essa sucessão ficasse em dúvida, um grupo de retentores \textit{bakufu} liderado por Hosokawa Fujitaka negociou com Nobunaga para obter apoio para o irmão mais novo de Yoshiteru, Yoshiaki.\\
            \\
            \indent Nobunaga, que havia se preparado durante um longo período para uma tal possibilidade, estabelecendo uma aliança com o clã Azai no norte da província de Omi e depois conquistando a província vizinha de Mino, agora marchava em direção a Kyoto. Após encaminhamento do clã Rokkaku no sul do Omi, Nobunaga forçou Matsunaga e Miyoshi a se retirar para Settsu. Ele então entrou na capital, onde ele ganhou reconhecimento do imperador para Yoshiaki, que se tornou o decimo quinto \textit{shogun} Ashikaga.\\
            \\
            \indent Nobunaga não tinha intenção de servir o \textit{bakufu} Muromachi, em vez disso voltou sua atenção para aumentar seu controle sobre a região Kinai. Resistência na forma de algum \textit{daimyo} rival, monges budistas intransigentes e comerciantes hostis foram eliminadas rapidamente e sem piedade, e Nobunaga ganhou a reputação de um adversário cruel e inflexível. Em apoio de seus movimentos políticos e militares, instituiu uma reforma econômica, removendo barreiras ao comércio por invalidar monopólios tradicionais detidos por santuários e guildas e instituindo o livre mercado conhecido como \textit{rakuichi-rakuza}.\\
            \\
            \indent Por volta de 1573, Oda Nobunaga havia destruído a aliança do clã Asakura e clãs Azai que ameaçava seu flanco norte, destruindo o centro monástico budista Tendai no Monte Hiei perto de Kyoto, e também havia conseguido evitar um confronto potencialmente perigoso, com Takeda Shingen, que tinha adoecido e morrido repentinamente quando o seu exército estava à beira de derrotar o Tokugawa e invadir os domínios de Oda a caminho de Kyoto.\\
            \\
            \indent Mesmo após a morte de Shingen, restavam vários \textit{daimyo} poderosos o suficiente para resistir a Nobunaga, mas nenhum estava situado perto o suficiente de Kyoto para representar alguma ameaça politica, e parecia que a unificação sob a bandeira Oda era uma questão de tempo.\\
            \\
            \indent Os inimigos de Nobunaga não eram apenas outras \textit{Sengoku daimyo} mas também adeptos da seita Jodo Shinshu do budismo que participaram dos \textit{Ikko-ikki}. \textit{Ikko-ikki} foram organizações de camponeses, monges budistas, padres e nobres locais, que se levantaram contra o domínio samurai entre os séculos quinze e dezesseis no Japão, seu líder era Kennyo. Ele suportou o ataque de Nobunaga contra sua fortaleza por dez anos. Nobunaga expulsou Kennyo no décimo primeiro ano, mas, por um tumulto causado por Kennyo, o território de Nobunaga recebeu um grande dano. Esta longa guerra foi chamado guerra Ishiyama Hongan-ji.\\
            \\
            \indent Para suprimir o Budismo, Nobunaga apoiou o cristianismo. Muitas culturas foram introduzidas ao Japão por missionários da Europa. A partir dessas culturas Japão recebeu novos alimentos, um novo método de desenho, astronomia, geografia, ciências médicas, e uma nova técnica de impressão.\\
            \\
            \indent Durante o período 1576-1579, Nobunaga construiu nas margens do Lago Biwa em Azuchi o castelo Azuchi, que ao contrário dos castelos anteriores não era para ser apenas uma estrutura militar, fria, escura. Nobunaga o construiu como uma mansão, que iria impressionar e intimidar seus rivais, não só com suas defesas, mas com os seus comodos luxuosos e uma bela decoração. A torre, chamada Tenshukaku, ao invés de ser o centro das defesas do castelo, era um edifício de sete andares com salas de audiência, aposentos privados, escritórios, e um tesouro, como se fosse um palácio real. Além de ser um dos primeiros castelos japoneses com uma torre, Azuchi foi o único que seu andar superior era octogonal. Sua fachada, ao contrário do branco ou preto sólido de outros castelos, foi vibrantemente decorada com tigres e dragões. Ele permaneceria como um símbolo da unificação.\\
            \\
            \indent Tendo assegurado o controle sobre a região Kinai, Nobunaga era agora poderoso o suficiente para atribuir a seus generais a tarefa de subjugar as províncias periféricas. A Shibata Katsuie foi dada a tarefa de conquistar o clã Uesugi em Etchu, Takigawa Kazumasu confrontou a provincia de  Shinano que um filho de Shingen, Takeda Katsuyori governava, e a Hideyoshi Hashiba foi dada a tarefa formidável de enfrentar o clã Mori na região de Chugoku, no oeste de Honshu.\\
            \\
            \indent Em 1576, Nobunaga obteve uma vitória significativa sobre o clã Takeda na batalha de Nagashino. Apesar da reputação da cavalaria de Takeda, Oda Nobunaga abraçou a tecnologia relativamente nova da espingarda, e infligiu uma derrota esmagadora. O legado desta batalha forçou uma revisão completa da guerra tradicional japonesa.\\
            \\
            \indent Em 1582, após uma campanha prolongada, Hideyoshi pediu ajuda Nobunaga em superar a vigorosa resistência. Nobunaga, fez uma parada em Kyoto, em sua viagem para o oeste com apenas um pequeno contingente de guardas, quando foi atacado por um de seus próprios generais descontentes, Akechi Mitsuhide. Que mesmo não o matando forçou Nobunaga a cometeu suicídio.\\
            
        \subsection {Toyotomi Hideyoshi}
             Toyotomi Hideyoshi nasceu como um camponês, no entanto ascendeu para finalmente terminar o Período Sengoku como um grande guerreiro. Na verdade, pouco se sabe ao certo sobre a carreira de Hideyoshi antes de 1570, o ano em que ele começa a aparecer em documentos e cartas que sobreviveram ao tempo. A autobiografia que ele encomendou começa no ano de 1577, além disso, Hideyoshi era conhecido por falar muito pouco sobre seu passado. Segundo a tradição, Hideyoshi nasceu em uma vila chamada Nakamura na província de Owari, filho de um camponês conhecido por nós como Yaemon. O nome de infância de Hideyoshi é registrado como Hiyoshimaru, ou "graça do sol", muito possivelmente um enfeite mais tarde inventado para dar substância a uma reivindicação de inspiração divina feita a respeito de seu nascimento. Na imagem popular da juventude de Hideyoshi ele é enviados para um templo, só para partir em busca de aventuras.\\
            \\
            \indent Após a morte de Nobunaga se seguiu uma disputa para decidir quem o vingaria e, consequentemente, estabeleceria uma posição dominante nas negociações sobre o eminente rearranjo do clã Oda. A situação se tornou ainda mais urgente quando se soube que o filho mais velho de Nobunaga e herdeiro, Nobutada, também foi morto, deixando o clã Oda sem sucessor claro.\\
            \\
            \indent Rapidamente começaram a negociar uma trégua com o clã Mori antes que eles pudessem saber da morte de Nobunaga, Hideyoshi tomou suas tropas em uma marcha forçada em direção ao seu adversário, a quem derrotou na batalha de Yamazaki, menos de duas semanas depois.\\
            \\
            \indent Apesar de um plebeu que se erguera através das fileiras de soldado de infantaria, Hideyoshi estava agora em posição de desafiar até mesmo o mais antigo dos possiveis heredeiros do clã Oda, e propôs que o filho, ainda bebê, de Nobutada, Sanposhi (que se tornou Oda Hidenobu), fosse nomeado herdeiro em vez do terceiro filho adulto de Nobunaga, Nobutaka, causa defendida por Shibata Katsuie. Tendo obtido o apoio de outros guerreiros, incluindo Niwa Nagahide e Tsuneoki Ikeda, Sanposhi foi nomeado herdeiro e Hideyoshi nomeado seu guardião.\\
            \\
            \indent A intriga política continuou, finalmente levando a um confronto aberto. Depois de derrotar Shibata na Batalha de Shizugatake em 1583 e passando por um impasse caro, mas em última instância vantajoso com Tokugawa Ieyasu na Batalha de Komaki e Nagakute em 1584, Hideyoshi conseguiu resolver a questão da sucessão uma vez por todas, assumindo o controle total de Kyoto, afim de se tornar o governante incontestável dos antigos dominios de Oda. O \textit{daimyo} de Shikoku, o clã Chosokabe, se rendeu a Hideyoshi em julho de 1585. Dois anos depois o clã Shimazu, \textit{daimyo} de Kyushu, também rendeu. Ele foi adotado pela família Fujiwara, dado o nome Toyotomi, e concedido o titulo de Kanpaku, que era em teoria uma espécie de conselheiro-chefe para o imperador, em razão do controle civil e militar de todo o Japão. No ano seguinte, ele havia assegurado alianças com três das nove grandes coligações \textit{daimyo} e levou a guerra de unificação até Shikoku e Kyushu. Em 1590, à frente de um exército de duzentos mil homens, Hideyoshi derrotou o clã Hojo, seu último rival no leste de Honshu. Os \textit{daimyo} restantes logo se renderam, e a reunificação militar do Japão estava completa.\\
            
        \subsection {O Japão Unificado}
            Com todo o Japão agora sob controle de Hideyoshi, uma nova estrutura para o governo nacional foi configurado. O país foi unificado sob um único líder, mas no dia-a-dia o governo do povo permaneceu descentralizado. A base do poder era a distribuição do território, medida pela produção de arroz em unidades de \textit{koku}. Em 1598, uma pesquisa nacional foi instituída e avaliou a produção nacional de arroz em 18,5 milhões \textit{koku}, dos quais 2 milhões era controlada diretamente pelo próprio Hideyoshi. Em contraste, Tokugawa Ieyasu, que Hideyoshi tinha transferido para a região de Kanto, mantinha 2,5 milhões \textit{koku}.\\
            \\
            \indent As pesquisas, realizadas por Hideyoshi antes e depois de conseguir o título de Taiko, são conhecidas como \textit{taiko kenchi}  (pesquisas taiko).\\
            \\
            \indent Uma série de outras inovações administrativas foram instituídas para incentivar o comércio e estabilizar a sociedade. A fim de facilitar o transporte, pedágios e postos de controle ao longo de estradas foram em grande parte eliminados, pois eram fortalezas militares desnecessárias. Medidas que efetivamente congelaram as distinções de classe foram instituídas, incluindo a exigência de que as diferentes classes vivessem separadamente em diferentes áreas de uma cidade e uma proibição dos agricultores transportarem ou possuirem armas. Hideyoshi ordenou a recolhimento de armas em uma grande \textit{katanagari} (caça de espadas).\\
            \\
            \indent Hideyoshi procurou assegurar a sua posição através de uma reorganização das participações dos \textit{daimyo} em sua vantagem. Em particular, ele transferiu a família Tokugawa para a região de Kanto, longe da capital, e cercou seu novo território com os vassalos mais confiáveis. Ele também adotou um sistema de reféns em que as esposas e herdeiros dos \textit{daimyo} residiam em sua cidade em Osaka.\\
            \\
            \indent Ele também tentou estabelecer uma sucessão ordenada, tomando o título de Taiko, ou "Kanpaku aposentado", em 1591 e tornando sucessor seu o sobrinho e filho adotivo Toyotomi Hidetsugu. Apenas mais tarde ele tentaria formalizar o equilíbrio de poder através da criação de órgãos administrativos. Estes incluíram o conselho dos cinco sábios, que foram empossados para manter a paz e apoiar a Toyotomi, o conselho de cinco membros de administradores, que tratou da política de rotina e questões administrativas, e da junta de três membros de mediadores, que foram encarregados de manter a paz entre os dois primeiros postos.\\
            \\
            \indent A última e principal ambição de Hideyoshi era conquistar a Dinastia Ming da China. Em abril de 1592, depois de ter sido recusada uma passagem segura através da Coréia, Hideyoshi enviou um exército de 200 mil homens para invadir e forçar sua passagem. Durante as invasões japonesas da Coréia (1592-1598), os japoneses ocuparam Seoul em maio de 1592, e com três meses de invasão chegoram a Pyongyang junto a um grande número de colaboradores coreanos, que no início viu os japoneses como libertadores da aristocracia corrupta. Rei Seonjo de Joseon fugiu, e dois príncipes coreanos foram capturados por Kato Kiyomasa. Seonjo enviou um emissário à corte Ming, pedindo ajuda militar urgentemente. O imperador chinês enviou almirante Chen Lin e o comandante Li Rusong para ajudar os coreanos. Li Rusong empurrou os japoneses até parte norte da península coreana. Os japoneses foram forçados a recuar até a parte sul da península coreana em janeiro de 1593, e começou o contra-atacou de Li Rusong. Este combate chegou a um impasse, até que Japão e China, finalmente começaram as negociações de paz.\\
			\\
			\indent Durante as negociações de paz que se seguiram entre 1593 e 1597, Hideyoshi, vendo o Japão como um igual de Ming, exigiu uma divisão da Coréia, condição de livre-comércio, e uma princesa chinesa como consorte para o imperador. Joseon e os líderes chineses não viram nenhuma razão para conceder a tais demandas, nem de tratar os invasores como iguais dentro do sistema de comércio Ming. Os pedidos do Japão foram negados e, assim, os esforços de paz chegaram a um beco sem saida.\\
			\\
			\indent A segunda invasão da Coréia começou em 1597, mas também resultou em fracasso, uma vez que as  forças japonesas se encontraram com um exercito coreano melhor organizado e um aumento do envolvimento chinês no conflito. Após a morte de Hideyoshi em 1598, as forças japonesas se retiraram da Coréia. A está altura o Conselho dos Cinco Sábios, e a maioria dos comandantes japoneses estavam mais preocupados com batalhas internas e para o controle do shogunato.\\
			\\
			\indent Em seu leito de morte Hideyoshi nomeou um grupo dos senhores mais poderosos do Japão - Tokugawa, Maeda, Ukita, Uesugi, Mori - para governar como o Conselho dos Cinco Sábios, até seu filho bebê, Hideyori, chegar a idade adulta. Uma paz precaria durou até a morte de Maeda Toshiie em 1599. Depois disso, Ishida Mitsunari acusou Ieyasu de ser desleal ao nome Toyotomi, precipitando uma crise que levou à Batalha de Sekigahara. Geralmente considerado como o último grande conflito do período Azuchi-Momoyama, a vitória de Ieyasu em Sekigahara marcou o fim do reinado Toyotomi. Três anos mais tarde, Ieyasu recebeu o título de \textit{Seii Taishogun}, e estabeleceu o \textit{bakufu} de Edo, que durou até a Restauração Meiji em 1868.\\
			
		\subsection{Nobunaga, Hideyoshi and Ieyasu}
			Toyotomi Hideyoshi, que unificou o Japão em 1590, e Tokugawa Ieyasu, que fundou o Shogunato Tokugawa em 1603, eram seguidores leais de Nobunaga. Estes dois foram agraciados com conquistas anteriores de Nobunaga em que eles poderiam construir um Japão unificado. Havia um ditado: "Nobunaga tritura o bolo de arroz nacional, Hideyoshi o amassa, e no final Ieyasu se senta e come".\\
			\\
			\indent Hideyoshi foi educado a partir de um camponês sem nome para ser um dos principais generais de Nobunaga. Quando ele se tornou um ministro em 1586, criou uma lei que a casta dos samurais foi codificada como permanente e hereditária, e os não-samurais foram proibidos de portar armas, terminando assim a mobilidade social do Japão a partir do qual ele mesmo havia se beneficiado. Estas restrições duraram até a dissolução do shogunato Edo pelos revolucionários na Restauração Meiji. Hideyoshi assegurou sua reivindicação como o legítimo sucessor de Nobunaga ao derrotar Mitsuhide Akechi um mês após a morte de Nobunaga.\\
			\\
			\indent É importante notar que a distinção entre o samurais e não-samurais era tão obscura que durante o século 16, a maioria dos adultos do sexo masculino em qualquer classe social (até mesmo os pequenos agricultores) pertenciam a pelo menos uma organização militar e serviu nas guerras antes e durante o governo de Hideyoshi. Pode-se dizer que uma situação de "todos contra todos" continuou durante um século. As famílias samurais autorizadas a partir do século 17 foram os que escolheram seguir Nobunaga, Hideyoshi e Ieyasu. Grandes batalhas ocorreram durante a mudança entre regimes e uma série de samurais derrotados foram destruídos, tornaram-se \textit{ronin} (samurais que não seguiam a um \textit{daimyo}) ou foram absorvidos pela população em geral.\\
			\\
			\indent Ieyasu tinha compartilhado sua infância com Nobunaga como um refém do clã Oda. Embora houvesse uma série de batalhas entre Ieyasu e o clã Oda, Ieyasu finalmente mudou de lado e se tornou um dos mais fortes aliados de Nobunaga.\\
    \newpage
    \begin{thebibliography}{4}
        \bibitem{turnbull96}
            \uppercase{turnbull}, Stephen.
            \emph{"The Samurai: A Military History"}.
            %Vol III: history from Oda Nobunaga to The Greater East War.
            %2nd Edition.
            Routledge; Reprint edition,
            1996.
        \bibitem{berry89}
            \uppercase{berry}, Mary Elizabeth.
            \emph{"Hideyoshi (Harvard East Asian Monographs)"}.
            %2nd Edition.
            Council on East Asian Studies Harvard University,
            1989.
        \bibitem{samuraiwiki}
            Samurai Wiki.
            \begin{verbatim}
				<http://wiki.samurai-archives.com/index.php?title=Oda_Nobunaga/>.
			\end{verbatim}
            %2nd Edition.
            Acesso em: 21 de novembro de 2011.
    \end{thebibliography}

\end{document}
