\documentclass[12pt, a4paper]{article}

\usepackage[brazil]{babel}
\usepackage[utf8]{inputenc}
\usepackage[T1]{fontenc}
\usepackage[pdftex]{hyperref}
\usepackage{graphicx}
\usepackage{amsmath}
%\usepackage{indentfirst}
\usepackage{fancyhdr}
\usepackage{setspace}

% Formatação
\topmargin -1.5cm
\oddsidemargin -0.04cm
\evensidemargin -0.04cm
\textwidth 16.59cm
\textheight 21.94cm 

%Cabeçalho e Rodape
%\pagestyle{empty}                     % Sem numero de paginas
%\pagestyle{fancy}                         %cabecalhos e rodares
%\fancyhead[RO,RE]{\today}
%\fancyhead[LO,LE]{MAC0332 - SI para grupos de pesquisa\\
%	Glossário}
%\fancyfoot[LO,LE]{Confidential}
%\fancyfoot[RO,RE] {\thepage}
%\fancyfoot[CO,CE]{Grupo 3, 2011}

\parskip 7.2pt                        % Espaço entre paragrafos 7.2
\renewcommand{\baselinestretch}{1.5} % Espaçamento entre linhas = 1.5
%\parindent 0pt

% Tirar hifenização
\hyphenpenalty = 5000
\tolerance = 1000
\sloppy

\title{Universidade De São Paulo\\
    Faculdade De Filosofia, Letras e Ciências Humanas\\
    Departamento De Línguas Orientais\\
    \vspace{3cm}
    \textbf{\Huge{Período Azuchi-Momoyama}}\\
    \vspace{4cm}}    
\author{Docente: Prof. Dr. Koichi Mori\\
    Discente: Geraldo Castro Zampoli\\
    Nº USP 6552380}
\date{Dezembro de 2011}
%\date{\today}

\begin{document}

    \maketitle
    \newpage
    
    \section {Processo de Unificação por Nobunaga e Hideyoshi}
        \subsection {Oda Nobunaga}
            Oda Nobunaga (1534 - 1582) foi um proeminente \textit{daimyo} (senhor feudal) do período Sengoku ao período Azuchi-Momoyama da história japonesa. Filho de um \textit{daimyo} menor da província de Owari, depois da morte de seu pai lutou contra outros membros de sua família para o controle do clã, matando um de seus irmãos no processo.\\
            \\
            \indent Durante a última metade do século 16, um número de diferentes \textit{daimyo} se tornoram fortes o suficiente seja para manipular o \textit{bakufu} Muromachi em proveito próprio ou derrubá-lo completamente. Uma tentativa de derrubar o \textit{bakufu} foi feita em 1560 por Imagawa Yoshimoto, cuja marcha para a capital chegou ao fim nas mãos de Oda Nobunaga na Batalha de Okehazama. Nesta batalha, Oda Nobunaga derrotou Imagawa Yoshimoto e se estabeleceu como um dos senhores da guerra no período Sengoku, também foi depois dessa batalha que Tokugawa Ieyasu da Província de Mikawa se juntou o Oda, sendo um fiel aliado deste momento até sua morte. Em 1562, o clã Tokugawa que estava ao leste do território de Nobunaga se tornou independente do clã Imagawa, e aliado com Nobunaga. A parte oriental do território de Nobunaga não foi invadida por esta aliança. Em 1565, uma aliança de Matsunaga e dos clãs Miyoshi tentou um golpe com o assassinato de Ashikaga Yoshiteru, o décimo terceiro \textit{shogun} Ashikaga. Disputas internas, no entanto, impediu-os de agir rapidamente para legitimar sua reivindicação de poder, somente em 1568 que conseguiram instalar o primo do Yoshiteru, Ashikaga Yoshihide, como o \textit{shogun} seguinte. O fracasso em entrar em Kyoto e ganhar reconhecimento da corte imperial, fez com que essa sucessão ficasse em dúvida, um grupo de retentores \textit{bakufu} liderado por Hosokawa Fujitaka negociou com Nobunaga para obter apoio para o irmão mais novo de Yoshiteru, Yoshiaki.\\
            \\
            \indent Nobunaga, que havia se preparado durante um longo período para uma tal possibilidade, estabelecendo uma aliança com o clã Azai no norte da província de Omi e depois conquistando a província vizinha de Mino, agora marchava em direção a Kyoto. Após encaminhamento do clã Rokkaku no sul do Omi, Nobunaga forçou Matsunaga e Miyoshi a se retirar para Settsu. Ele então entrou na capital, onde ele ganhou reconhecimento do imperador para Yoshiaki, que se tornou o decimo quinto \textit{shogun} Ashikaga.\\
            \indent Nobunaga não tinha intenção de servir o \textit{bakufu} Muromachi, em vez disso voltou sua atenção para aumentar seu controle sobre a região Kinai. Resistência na forma de algum daimyo rival, monges budistas intransigentes e comerciantes hostis foram eliminadas rapidamente e sem piedade, e Nobunaga rapidamente ganhou uma reputação como um adversário cruel e inflexível. Em apoio de seus movimentos políticos e militares, instituiu uma reforma econômica, removendo barreiras ao comércio por invalidar monopólios tradicionais detidos por santuários e guildas e instituindo o livre mercado conhecido como \textit{rakuichi-rakuza}.\\
            \\



            %Em 1560, enfrentou um grande exército (estimado em 40 mil soldados), comandado por Imagawa Yoshimoto, com apenas 3.000 soldados durante a batalha de Okehazama. Graças a um ataque surpresa foi vitorioso, colocando Nobunaga no topo do poder militar do país.\\
            %\indent Em 1568 ajudou Ashikaga Yoshiaki a ser nomeado Shogun pelo imperador, entrando no capital, Kyoto, com seu exército e assumindo o controle da cidade. Yoshiaki queria nomeá-lo Kanrei (alto posto politico no Japão feudal), mas ele se recusou e em vez disso aprovou uma série de regulamentações, que limitavam a atividade do shogun praicamente a assuntos cerimoniais. Yoshiaki então contatou vários daimyo e monges guerreiros para formar uma coalizão contra Nobunaga, que se encontrou com eles entre 1570 e 1573, ano em que a rivalidade entre o shogun e Nobunaga foi tornada pública e aberta. Nobunaga enfrentou Yoshiaki e o derrotou facilmente, terminando o shogunato Ashikaga.\\
            
        \subsection {Toyotomi Hideyoshi}
            
        \subsection {O Processo de Unificação}
            
    \newpage
    \begin{thebibliography}{9}
        \bibitem{hiraizumi02}
            \uppercase{kiyoshi}, Hiraizumi.
            \emph{"The history of Japan"}.
            Vol III: history from Oda Nobunaga to The Greater East War.
            %2nd Edition.
            Kyoto: Sesei Kihaku,

            2002.
        \bibitem{yamashiro86}
            \uppercase{yamashiro}, José.
            \emph{"História da Cultura Japonesa"}.
            %2nd Edition.
            São Paulo: IBRASA,
            1986.
        \bibitem{varley94}
            \uppercase{valey}, Paul.
            \emph{"Warriors of Japan"}.
            %2nd Edition.
            University of Hawaii Press,
            1959.
        \bibitem{mccullough59}
            \uppercase{mccullough}, Helen Craig.
            \emph{"The Taiheiki: A Chronicle of Medieval Japan"}.
            %2nd Edition.
            Charles E. Tuttle Company, Tokyo,
            2002.
    \end{thebibliography}

\end{document}
